\begin{problem}[1.17]
温度为$0^\circ C$的1kg水与温度为$100^\circ C$的恒温热源接触后, 水温达到
$100^\circ C$. 试分别求水和热源的熵变以及整个系统的总熵变. 欲使参与过程的整
个系统的熵保持不变, 应如何使水温从$0^\circ C$升至$100^\circ C$? 已知水的比热容为
4.18$\mathrm{J\cdot g^{-1} K^{-1}}$.
\end{problem}
% --------------------------------------------------------------------
\begin{solution}
温度为$0^\circ C$的1kg水与温度为$100^\circ C$的恒温热源接触后, 水温达到
$100^\circ C$, 水的熵变为
\[
\Delta S_{\text{水}} = \int_{0+273.15}^{100+273.15} \frac{c_p m d T}{T} = c_p m \ln\frac{373.15}{273.15} = 1304.0 \mathrm{~J/K}
\]
恒温热源放出的热量等于水吸收的热量, 因此热源熵变
\[
\Delta S_{\text{源}} = -\frac{Q}{T_{\text{源}}} = -\frac{c_p m \Delta T}{373.15} = -1120.2 \mathrm{~J/K}
\]
因此系统的总熵变为
\[
\Delta S_{\text{总}} = \Delta S_{\text{水}} + \Delta S_{\text{源}} = 183.8 \mathrm{~J/K}
\]

欲使参与过程的整个系统的熵保持不变, 并使水温从$0^\circ C$升至$100^\circ C$, 则该过程必需是可逆过程: 让水分别从温度在$0^\circ C$到$100^\circ C$之间的一系列热源吸热. 此时热源的熵变为
\[
\Delta S_{\text{源}}=-\int_{273.15}^{373.15} \frac{c_p m d T}{T}  = -\Delta S_{\text{水}}
\]
因此整个系统的熵保持不变.
\end{solution}
