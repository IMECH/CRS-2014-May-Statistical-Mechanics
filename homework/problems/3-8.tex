\begin{problem}[3.8]
在三相点附近, 固态氨的蒸气压(单位为Pa)方程为
\[
\ln p = 27.92 - 3754/T
\]
液态氨的蒸气压力方程为
\[
\ln p = 24.38 - 3063/T
\]
试求氨三相点的温度和压强, 氨的汽化热, 升华热及在三相点的熔解热.
\end{problem}
% --------------------------------------------------------------------
\begin{solution}
固态氨的蒸气压方程是固相与气相的两相平衡曲线, 液态氨的蒸气压方程是液相与气想的两相平衡曲线. 氨三相点的温度$T_t$可由这两条曲线的交点确定
\[
27.92 - 3754/T_t = 24.38 - 3063/T_t \quad\Longrightarrow
T_t = 195.2 ~\mathrm{K}
\]
将$T_t = 195.2 ~\mathrm{K}$代入蒸气压方程得
\[
p_t = 5934 ~\mathrm{Pa}
\]
由蒸气压方程及式(3.4.8)$\ln p = -\frac{L}{RT}+A$可得得汽化热$L_\text{v}$和升华热$L_\text{s}$
\[
L_\text{v} = 3.120\times 10^4 ~\mathrm{J}, \qquad
L_\text{s} = 2.547\times 10^4 ~\mathrm{J}
\]
氨在三相点的熔解热$L_{溶}$等于
\[
L_\text{f} = L_\text{v}- L_\text{s} = 0.573\times 10^4 ~\mathrm{J}
\]
\end{solution}
