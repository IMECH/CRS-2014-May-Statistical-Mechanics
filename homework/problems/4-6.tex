\begin{problem}[4.6]
如图4.6所示, 开口玻璃管底端有半透膜将管中的糖的水溶液与容器内的水隔开. 半透膜只让水透过, 不让糖透过. 实验发现, 糖水溶液的液面比容器内的水现上升一个高度$h$, 表明在同样温度下糖水溶液的压强$p$与水的压强$p_0$之差为$p - p_0 = \rho g h$.
这一压强差称为渗透压. 试证明, 糖水与水达到平衡时有
\[
g_1(T,p)+RT\ln(1-x) = g_1(T,p_0)
\]
其中$g_1$是纯水的摩尔吉布斯函数, $x$是糖水中糖的摩尔分数, $x=\frac{n_2}{n_1+n_2}\approx \frac{n_2}{n_1}\ll 1$. 试据此证明
\[
p - p_0=\frac{n_2RT}{V}
\]
其中$V$是糖水溶液的体积.
\end{problem}
% --------------------------------------------------------------------
\begin{solution}
管中的糖水和容器内的水形成两相. 平衡时两相的温度必须相等. 由于水可以通过半透膜, 水在两相中的化学势也必须相等. 半透膜可以承受两边的压强差,两相的压强不必相等. 以$p$表示管内糖水的压强,$p_0$表示容器内纯水的压强. 根据$\mu_i(T,p)=g_i(T,p)+RT\ln x_i^L$, 管内糖水中水的化学势为
\[
\mu_1(T,p) = g_1(T,p) + RT\ln(1-x)
\]
容器内纯水的化学势为$g_1(T,p_0)$. 相平衡条件要求
\begin{equation}\label{eq:4-6g1RT}
g_1(T,p)+RT\ln(1-x) = g_1(T,p_0)
\end{equation}
由于$p$和$p_0$相差很小, 可令
\[
g_1(T,p)-g_1(T,p_0) = \bigg(\frac{\partial g_1}{\partial p}\bigg)_T(p-p_0) = V_{1m}(p-p_0)
\]
其中用了$\d G=-S\d T + V\d p$. $V_{1m}$是纯水的摩尔体积. 代入式(\ref{eq:4-6g1RT})得
\begin{equation}\label{eq:4-6pp0}
p - p_0 =-\frac{RT}{v_{1m}}\ln(1-x)
\end{equation}
在$x\ll 1$的情形下, 可以作近似
\[
\ln(1-x) \approx -x
\]
且糖水溶液的体积 $v\approx n_1V_{1m}$, 因此式(\ref{eq:4-6pp0})可近似为
\[
p - p_0 = \frac{RT}{V_{1m}} x = \frac{RT}{v_{1m}}\frac{n_2}{n_1} = \frac{n_2RT}{V}
\]
\end{solution}
