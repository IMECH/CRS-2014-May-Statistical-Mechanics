\section{涨落理论}

% ---------------------------------------

\subsection{涨落的准热力学理论}
宏观量是相应微观量在满足给定宏观条件的系统的所有可能的微观状态上的平均值. 系
统处于微观状态的微观量与平均值的差别即为涨落. 涨落的准热力学理论给出了给定宏
观条件下热力学量取各种涨落值的概率分布.

系统的能量和体积对最概然值具有偏差$\Delta E$和$\Delta V$的概率与$\Omega^{(0)}$成正比
$W\propto \exp\{\Delta S^{(0)}/k\}$:
\begin{itemize}
\item \textbf{正则系综}:
\[
W\propto \exp\Bigg\{-\frac{\Delta E-T\Delta S+p\Delta V}{kT}\Bigg\}
\]
\[
W\propto \exp\Bigg\{-\frac{\Delta S\Delta T-\Delta p\Delta V}{2kT}\Bigg\}
\]
\[
W\propto \exp\Bigg\{-\frac{C_v}{2kT^2}(\Delta T)^2 + \frac{1}{2kT}\bigg(\frac{\partial p}{\partial V}\bigg)_T(\Delta V)^2 \Bigg\}
\]
\item \textbf{巨正则系综}:
\[
W\propto \exp\Bigg\{-\frac{\Delta S\Delta T-\Delta p\Delta V +\Delta\mu\Delta N}{2kT}\Bigg\}
\]
\[
W\propto \exp\Bigg\{-\frac{C_v}{2kT^2}(\Delta T)^2 + \frac{1}{2kT}\bigg(\frac{\partial \mu}{\partial N}\bigg)_{V,T}(\Delta N)^2 \Bigg\}
\]
\end{itemize}
% ---------------------------------------

\subsection{朗之万方程和爱因斯坦关系}
\begin{itemize}
\item\textbf{朗之万方程}: 布朗运动颗粒的运动方程
\[
m\frac{\d^2 x}{\d t^2} = -\alpha \frac{\d x}{\d t} + F(t) + \underbrace{f(t)}_{\textrm{外力项}}.
\]
其中$\alpha=6\pi a \eta$.
\item\textbf{爱因斯坦关系}: 爱因斯坦关系给出了温度为$T$时颗粒在介质中的粘滞阻力系数$\alpha$与扩散系数$D$的关系
\[
D = \frac{kT}{\alpha}
\]
\end{itemize}

% ---------------------------------------
