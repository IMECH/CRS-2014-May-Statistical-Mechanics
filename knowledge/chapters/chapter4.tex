\section{多元系的复相平衡和化学平衡热力学第三定律}

% ---------------------------------------

\subsection{多元系的热力学方程}
\[
\d G = -S\d T + V\d p + \sum_i\mu_i\d n_i 
\]
\[
\d U = T\d S + p\d V + \sum_i\mu_i\d n_i 
\]
\[
S\d T - V\d p + \sum_i\mu_i\d n_i = 0
\]

其中$X=\{V, U, S\}$可表示为$X= \sum_i(n_i\partial X/\partial n_i)_{T,p,n_j}$.
% ---------------------------------------

\subsection{多元系的相变平衡条件}
系统达到平衡时,两相中各组员的化学势必分别相
等: $\mu_i^\alpha = \mu_i^\beta$

% ---------------------------------------

\subsection{吉布斯相律}
若多元系有$\phi$个相, 每相有$k$个组元, 则多元复相的自由度数$f$有
\[
f = k + 2 - \phi
\]
% ---------------------------------------

\subsection{化学平衡条件}
若在反应方程中$i$组元的系数为$\nu_i$, 化学势为$\mu_i$. 则化学平衡条件为:
\[
\sum_i \nu_i \mu_i = 0
\]
% ---------------------------------------

\subsection{混合理想气体的性质}
\begin{itemize}
\item \textbf{道尔顿分压定律}: 混合气体压强等于各组元分压之和: $p=\sum_ip_i$, 其中的$p_i$为$i$组元的分压.
\item \textbf{混合理想气体物态方程}: $pV=\sum_{i}n_iRT$
\item \textbf{混合理想气体吉布斯函数}
\[
G=\sum_in_i\mu_i=\sum_in_iRT\Big[\phi_i+\ln(x_ip)\Big]
\]
\item \textbf{混合理想气体的熵}
\[
S = \sum_i n_i\bigg[\int c_{pi}\frac{\d T}{T}-R\ln(x_i,p) + s_{i0}\bigg]
\]
\item \textbf{混合理想气体的焓}
\[
H = \sum_i n_i\bigg[\int c_{pi}\d T + h_{i0}\bigg]
\]
\end{itemize}

% ---------------------------------------

\subsection{理想气体的化学平衡(平衡常量)}
平衡常量$K$及定压平衡常量$K_p$
\[
K(T,p) = p^{-\nu}K_p, \quad (\nu=\sum_i v_i)
\]
\[
\ln K_p = -\sum_i \nu_i\phi_i(T)
\]

% ---------------------------------------

\subsection{热力学第三定律}
不可能通过有限的步骤使一个物体冷却到绝对温度的零度.
