\section{玻尔兹曼统计}

% ---------------------------------------

\subsection{热力学量的统计表达式}
\begin{itemize}
\item\textbf{量子统计}: 对于配分函数为
\[
z_1 =\sum_l \omega_l \mathrm{e}^{-\beta \varepsilon_l}, \quad N= \mathrm{e}^{-\alpha} Z_1
\]
则\textbf{内能}, \textbf{广义作用力}, \textbf{熵}的统计表达式分别为
\[
U = -N\frac{\partial}{\partial\beta}\ln Z_1 
\]
\[
F = \textcolor{red}{-}\frac{N}{\beta}\frac{\partial}{\partial y}\ln Z_1 ,\quad \textrm{如} p = \textcolor{blue}{+}\frac{N}{\beta}\frac{\partial}{\partial y}\ln Z_1 
\]
\[
\underbrace{S = k\ln\Omega}_{\textrm{玻尔兹曼关系}} = Nk\bigg( \ln Z_1 -\beta \frac{\partial}{\partial} \ln Z_1 \bigg) 
\]
对于满足经典极限条件的玻色(费米)系统: $S=k\ln(\Omega_{M.B.}/N!)$.  自由能定域系统: $F=-NkT\ln Z_1$.
满足极限条件的玻色/费米系统:  $F=-NkT\ln Z_1 + kT\ln N!$.

\item\textbf{经典统计}:配分函数和波尔兹曼分布经典表达式
\[
Z_1=\sum_l \mathrm{e}^{-\beta \varepsilon_l}\frac{\Delta \omega_l}{h_0^r}
\]
\[
a_l = \mathrm{e}^{-\alpha -\beta\varepsilon_l}\frac{\Delta \omega_l}{h_0^r}
=\frac{N}{Z_1}\mathrm{e}^{-\beta\varepsilon_l}\frac{\Delta \omega_l}{h_0^r}
\]
\end{itemize}
% ---------------------------------------

\subsection{理想气体的物态方程表达式, 经典极限条件}
\begin{itemize}
\item\textbf{理想气体的物态方程}:
\[
p = \frac{N}{\beta}\frac{\partial}{\partial V}\ln Z_1 = \frac{NkT}{V}, 
\quad
Z_1 = V\bigg(\frac{2\pi\m}{\hbar^2\beta}\bigg)^{3/2}
\]
\item\textbf{经典极限条件}:
\[
\mathrm{e}^\alpha = \frac{V}{N}\bigg(\frac{2\pi mkT}{h^2}\bigg)^{3/2} \gg 1
\quad\textrm{或}\quad
n\lambda^3\ll 1
\]
\end{itemize}

% ---------------------------------------


\subsection{麦克斯韦速度分布率、最概然速率}
\begin{itemize}
\item\textbf{麦克斯韦速度分布率}: 
%\[
%f(v_x,v_y,v_z)= n\bigg(\frac{m}{2\pi kT}\bigg)^{3/2}\exp\Big\{-\frac{m}{2kT}(v_x^2+v_y^2+v_z^2)\Big\}
%\]
\[
f(v_{x},v_{y},v_{z})=n\bigg(\frac{m}{2\pi kT}\bigg)^{3/2}\mathrm{e}^{-\frac{m}{2kT}(v_{x}^{2}+v_{y}^{2}+v_{z}^{2})}
\]
萁中$n=N/V$为单位体积内的分子数, $m$为质量. 则速度在$\d v_x\d v_y\d v_z $内的分子数可表示为$f(v_x,v_y,v_z)\d v_x\d v_y\d v_z $.
\item\textbf{最概然速率}: 使速度分布函数取最大值的速率
\[
v_{m} = \sqrt{2kT/m}
\]
\item\textbf{平均速率}:
\[
\overline{v} = \sqrt{8kT/(\pi m)}
\]
\item\textbf{方均根速率}:
\[
v_{s} = \sqrt{3kT/m}
\]
\end{itemize}

% ---------------------------------------

\subsection{能量均分定律}
对于处在温度为$T$的平衡状态的经典系统, 粒子能量中每个平方项的平均值等于$kT/2$.

\subsection{气/固体的内能, 定容和定压热容量}
\begin{itemize}
\item\textbf{单原子分子}:
\[
U = \frac{3}{2}NkT,\quad
C_v = \frac{3}{2}Nk,\quad
C_p = \frac{5}{2}Nk
\]
\item\textbf{双原子分子}:
\[
U = \frac{5}{2}NkT,\quad
C_v = \frac{5}{2}Nk,\quad
C_p = \frac{7}{2}Nk
\]
\item\textbf{固体分子}:
\[
U = 3NkT,\quad
C_v = 3Nk,\quad
C_p = C_v + \frac{TVa^2}{K_T}
\]
\end{itemize}

% ---------------------------------------

\subsection{理想气体的内能和热容量}
\[
U = -N\frac{\partial}{\partial\beta}\ln Z_1 = 
-N\frac{\partial}{\partial\beta}\big(
\underbrace{\ln Z_1^\mathrm{t}}_{\textrm{平动}}
+\underbrace{\ln Z_1^\mathrm{v}}_{\textrm{振动}}
+\underbrace{\ln Z_1^\mathrm{r}}_{\textrm{转动}}
\big)
\]
\[
C_v = \underbrace{C_v^\mathrm{t}}_{\textrm{平动}}
+\underbrace{C_v^\mathrm{v}}_{\textrm{振动}}
+\underbrace{C_v^\mathrm{r}}_{\textrm{转动}}
\]

% ---------------------------------------

\subsection{单原子理想气体的熵}
\[
S = \frac{3}{2}Nk\ln T + Nk\ln \frac{V}{N} + \frac{3}{2}Nk\Bigg[\frac{5}{3}+\ln\bigg(\frac{2\pi mk}{h^2}\bigg)\Bigg]
\]

% ---------------------------------------

\subsection{固体的热容量}
\[
C_v = 3Nk\bigg(\frac{\theta_E}{T}\bigg)^2 \frac{\exp\{\theta_E/T\}}{\big(\exp\{\theta_E/T\}-1\big)^2}
\]
爱因斯坦特征温度: $k\theta_E = \hbar \omega$. 当$T\gg\theta_E$, 上式可近似为$C_v=3Nk$; 当$T\ll\theta_E$上式可近似为$C_v=3Nk(\theta_E/T)^2\exp\{-\theta_E/T\}$.

